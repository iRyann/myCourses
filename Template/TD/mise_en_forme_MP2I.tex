%%%%%%%%%%%%%%%%%%%%%%%%%%%%%%%%%%%%%%%%%%%%
% MISE EN FORME DU TITRE 
%%%%%%%%%%%%%%%%%%%%%%%%%%%%%%%%%%%%%%%%%%%%
\newcommand{\titreCopie}[1]{
\noindent
%\\%[0.4cm]
\begin{center}
\LARGE
\hrule
\vspace{.4cm}
#1 \\[0.4cm]
\hrule
\end{center}

\vspace*{0.4cm}
}


%%%%%%%%%%%%%%%%%%%%%%%%%%%%%%%%%%%%%%%%%%%%
% MISE EN FORME DES PIEDS DE PAGE
%%%%%%%%%%%%%%%%%%%%%%%%%%%%%%%%%%%%%%%%%%%%
\AtEndDocument{\label{lastpage}}
\lhead{}
\chead{}
\rhead{}
\lfoot{ 
Ryan Bouchou
}
\cfoot{}
\rfoot{\thepage/\pageref{lastpage}}
\renewcommand{\headrulewidth}{0pt}
\fancyhfoffset{0.7cm}



%%%%%%%%%%%%%%%%%%%%%%%%%%%%%%%%%%%%%%%%%%%%
% REMISE EN FORME DES SOUS-SOUS-SECTIONS  
%%%%%%%%%%%%%%%%%%%%%%%%%%%%%%%%%%%%%%%%%%%%

% section = exercices
\renewcommand{\thesection}{Exercice~\arabic{section}}
% sous-section = question
\renewcommand{\thesubsection}{Question~\arabic{subsection}}




%%%%%%%%%%%%%%%%%%%%%%%%%%%%%%%%%%%%%%%%%%%%
% MISE EN FORME DU CODE AVEC MINTED
%%%%%%%%%%%%%%%%%%%%%%%%%%%%%%%%%%%%%%%%%%%%

%pour le style des numéros de ligne minted
\renewcommand{\theFancyVerbLine}{\sffamily
\textcolor{expli}{\scriptsize
\oldstylenums{\arabic{FancyVerbLine}}}}

%pour éviter l'italique 
%car apparaissaient en italique à la fois les commentaires et les includes
\newtoggle{inminted}
\AtBeginEnvironment{minted}{\let\itshape\relax}

%paramètres pour le c
\setminted[c]{
	%--fond et cadre	
	%bgcolor = black!3!white,
	frame = leftline,
	framesep = 6pt,
	rulecolor= expli,
	%--numéro de ligne
	linenos=true,
	numbersep=4pt,
	%stepnumber=2,
	xleftmargin=20pt,%pour que les chiffres débordent pas
	%--découpage des longues lignes
	breaklines=true,
	%--tabulations
	tabsize=2,
	}

%paramètres pour le ocaml
\setminted[ocaml]{
	%--fond et cadre	
	%bgcolor = black!3!white,
	frame = leftline,
	framesep = 6pt,
	rulecolor= expli,
	%--numéro de ligne
	linenos=true,
	numbersep=4pt,
	%stepnumber=2,
	xleftmargin=20pt,%pour que les chiffres débordent pas
	%--découpage des longues lignes
	breaklines=true,
	%--tabulations
	tabsize=2,
	}


