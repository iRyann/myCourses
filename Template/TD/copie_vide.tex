\documentclass[12pt,a4paper,fleqn]{article}

%%%%%%%%%%%%%%%%%%%%%% LES PACKAGES
\input{packages_MP2I.tex}

%%%%%%%%%%%%%%%%%%%%%% MISE EN FORME
%%%%%%%%%%%%%%%%%%%%%%%%%%%%%%%%%%%%%%%%%%%%
% MISE EN FORME DU TITRE 
%%%%%%%%%%%%%%%%%%%%%%%%%%%%%%%%%%%%%%%%%%%%
\newcommand{\titreCopie}[1]{
\noindent
%\\%[0.4cm]
\begin{center}
\LARGE
\hrule
\vspace{.4cm}
#1 \\[0.4cm]
\hrule
\end{center}

\vspace*{0.4cm}
}


%%%%%%%%%%%%%%%%%%%%%%%%%%%%%%%%%%%%%%%%%%%%
% MISE EN FORME DES PIEDS DE PAGE
%%%%%%%%%%%%%%%%%%%%%%%%%%%%%%%%%%%%%%%%%%%%
\AtEndDocument{\label{lastpage}}
\lhead{}
\chead{}
\rhead{}
\lfoot{ 
Ryan Bouchou
}
\cfoot{}
\rfoot{\thepage/\pageref{lastpage}}
\renewcommand{\headrulewidth}{0pt}
\fancyhfoffset{0.7cm}



%%%%%%%%%%%%%%%%%%%%%%%%%%%%%%%%%%%%%%%%%%%%
% REMISE EN FORME DES SOUS-SOUS-SECTIONS  
%%%%%%%%%%%%%%%%%%%%%%%%%%%%%%%%%%%%%%%%%%%%

% section = exercices
\renewcommand{\thesection}{Exercice~\arabic{section}}
% sous-section = question
\renewcommand{\thesubsection}{Question~\arabic{subsection}}




%%%%%%%%%%%%%%%%%%%%%%%%%%%%%%%%%%%%%%%%%%%%
% MISE EN FORME DU CODE AVEC MINTED
%%%%%%%%%%%%%%%%%%%%%%%%%%%%%%%%%%%%%%%%%%%%

%pour le style des numéros de ligne minted
\renewcommand{\theFancyVerbLine}{\sffamily
\textcolor{expli}{\scriptsize
\oldstylenums{\arabic{FancyVerbLine}}}}

%pour éviter l'italique 
%car apparaissaient en italique à la fois les commentaires et les includes
\newtoggle{inminted}
\AtBeginEnvironment{minted}{\let\itshape\relax}

%paramètres pour le c
\setminted[c]{
	%--fond et cadre	
	%bgcolor = black!3!white,
	frame = leftline,
	framesep = 6pt,
	rulecolor= expli,
	%--numéro de ligne
	linenos=true,
	numbersep=4pt,
	%stepnumber=2,
	xleftmargin=20pt,%pour que les chiffres débordent pas
	%--découpage des longues lignes
	breaklines=true,
	%--tabulations
	tabsize=2,
	}

%paramètres pour le ocaml
\setminted[ocaml]{
	%--fond et cadre	
	%bgcolor = black!3!white,
	frame = leftline,
	framesep = 6pt,
	rulecolor= expli,
	%--numéro de ligne
	linenos=true,
	numbersep=4pt,
	%stepnumber=2,
	xleftmargin=20pt,%pour que les chiffres débordent pas
	%--découpage des longues lignes
	breaklines=true,
	%--tabulations
	tabsize=2,
	}




%%%%%%%%%%%%%%%%%%%%%%% LA PALETTE DE COULEURS
\input{palette_aef.tex}

%%%%%%%%%%%%%%%%%%%%%%% LES COMMANDES 
\input{commandes_MP2I.tex}

\PassOptionsToPackage{svgnames}{xcolor}

%%%%%%%%%%%%%%%%%%%%%%%%%%%%%%%%%%%%%%%
\begin{document}
\pagestyle{fancy} %active les pieds de pages
\tcbset{colback=red!5!white,colframe=red!75!black,fonttitle=\bfseries}


%%%%%%%%%%%%%%%%%%%%%%%%%%%%%%%%%%%%%%%
\titreCopie{TD Maths}
%%%%%%%%%%%%%%%%%%%%%%%%%%%%%%%%%%%%%%%
\section{Calcul littéral}
\subsection{Simplification d'expressions}
\begin{enumerate}
    \item \(3x + 2y - 5x + 4y\)
    \item \(2a - 3b + 5a + b\)
    \item \(4m - 2n + 7m - 3n\)objet
\end{enumerate}
\subsection{Factorisation}
\begin{enumerate}
    \item  \(6x + 9y\)
    \item  \(15a - 5b\)
    \item  \(10m^2 - 5n\)
\end{enumerate}
\subsection{Développement}
\begin{enumerate}
    \item \((x + 3)(x - 2)\)
    \item \((2a - 5)(3a + 4)\)
    \item \((3m + 2)(m - 1)\)
\end{enumerate}
\subsection{Résolution d'équations}
\begin{enumerate}
    \item \(2x + 5 = 3x - 1\)
    \item \(4(x - 2) = 12\)
    \item \(2(3y + 1) = 5y - 3\)
\end{enumerate}
\subsection{Problème}
Un rectangle a une aire $A$ donnée par \(A = 3x^2 + 5x - 2\). Si la longueur du rectangle est \(x + 2\), quelle est l'expression pour la largeur du rectangle ? Quelle est la largeur si $x=3$ ?

\section{Dérivation}
\subsection{Dérivées de fonctions simples}
\begin{enumerate}
 \item \(f(x) = 3x^2 - 4x + 1\)
 \item \(g(x) = \sqrt{x} + 2x\)
 \item \(h(x) = e^x - \ln(x)\)
\end{enumerate}
\subsection{Règles de dérivation}
\begin{enumerate}
    \item  \(f(x) = 4x^3 + 2x^2 - 5x + 1\)
    \item  \(g(x) = \frac{2}{x} + \sin(x)\)
    \item  \(h(x) = e^{2x} \cdot \cos(x)\)
\end{enumerate}
\subsection{Dérivées successives}
\begin{enumerate}
    \item \(f(x) = x^4 - 2x^3 + 3x^2 - x + 1\)
    \item  \(g(x) = \frac{1}{x^2} + \cos(x)\)
\end{enumerate}
\subsection{Problème}
\subsubsection{} La fonction de coût d'une entreprise est donnée par \(C(x) = 2x^2 + 3x + 5\), où \(x\) est la quantité produite. Trouver la dérivée de la fonction coût par rapport à \(x\).
\subsubsection{} La position d'une particule est donnée par la fonction \(s(t) = 4t^2 - 3t + 1\), où \(t\) est le temps en secondes. Trouver la vitesse de la particule en fonction du temps en calculant la dérivée de \(s(t)\).
\subsection{}
Soit \(f(x) = x^3 - 3x^2 + 2x + 5\). Trouver les points où la tangente à la courbe de \(f(x)\) est horizontale.
\section{Fonction $e^x$ et $ln$}
\subsection{Exponentielle}
\begin{enumerate} 
\item  Soit \(f(x) = 2^x\). Calculer \(f(0)\), \(f(1)\), et \(f(2)\).
\item  Résoudre l'équation \(2^x = 8\).
\item  Simplifier l'expression \(\frac{2^{x+1}}{2^x}\).
\end{enumerate}
\subsection{Logarithme}
\begin{enumerate}
\item Résoudre l'équation \(\log_2(x) = 3\).
\item Simplifier l'expression \(\log_3(9) - \log_3(3)\).
\item Si \(\log_a(b) = 2\), quelle est la valeur de \(a\)?
\end{enumerate}
\subsection{Applications}
\subsubsection{} 
La population d'une ville suit une croissance exponentielle de \(P(t) = 1000 \times 1.02^t\), où \(t\) est le nombre d'années depuis aujourd'hui. Quelle sera la population dans 5 ans?
\subsubsection{} Un investissement initial de 5000 € augmente de 8\% chaque année. Modélisez la valeur de l'investissement en fonction du temps avec une fonction exponentielle.
\subsubsection{} Une substance radioactive perd 10\% de son activité chaque année. Si l'activité initiale est de 1000 Bq, quelle sera l'activité après 3 ans?
\subsection{Plusieurs fonctions}
\begin{enumerate}
    \item  Soit \(g(x) = e^{2x}\) et \(h(x) = \ln(x)\). Calculer \(g(1)\) et \(h(e)\).
    \item  Résoudre l'équation \(e^{2x} = 16\).
\end{enumerate}
\subsection{}
\begin{enumerate}
    \item  Représenter graphiquement la fonction \(y = e^x\) sur l'intervalle \(\mathbb{R}\) ainsi que les éventuelles asymptotes, et la tangeante en 0.
\item  Déterminer la ou les asymptotes de la fonction \(y = \ln(x)\).
\item  Soit \(k(x) = \frac{e^{2x} - 1}{e^x + 1}\). Simplifier \(k(x)\).
$$log_b(x)=\frac{ln(x)}{ln(b)}$$
\end{enumerate}
\end{document}

