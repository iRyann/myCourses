%%%%%%%%%%%%%%%%%%%%%%%%%%%%%%%%%%%%%%%%%%%%
% GENERAL  
%%%%%%%%%%%%%%%%%%%%%%%%%%%%%%%%%%%%%%%%%%%%

%raccourcis généraux
\newcommand {\ie}{\textit{i.e}. }
\newcommand {\cf}{\textit{Cf.} }
\newcommand {\resp}{resp.\! }


%pour les belles fonctions (à mettre entre $ $  ou entre $$ $$ )
\newcommand{\fonction}[5]{ 
#1 =
\left( \! \begin{tabular}{c@{ }c@{ }l} 
$#2$  & $\longrightarrow$   & $#3$  \\
$#4$  & $\longmapsto$       & $#5$ 
\end{tabular} \! \right)
}

%le widebar (inutile si package mathabx
\newcommand{\widebar}[1]{\overline{#1}}

%symboles rapprochés
\renewcommand {\=}{\!=\!}
\renewcommand {\-}{\!-\!}
\renewcommand {\:}{\!:\!}
\newcommand {\+}{\!+\!}
\newcommand {\iin}{\!\in\!}
\newcommand {\inclus}{\!\subseteq\!}

%symboles
\newcommand{\bubullet}{-}
\newcommand{\subbullet}{\circ}
\newcommand{\parties}{\mathcal{P}}

%produit scalaire
\renewcommand{\.}{\!\cdot\!}

%indicatrice
\newcommand{\1}{\mathbb{I}}

%cardinal
\newcommand{\card}[1]{\left|#1\right|}

%%%%%%%%%%%%%%%%%%%%%%%%%%%%%%%%%%%%%%%%%%%%
% ENSEMBLES DE NOMBRES & ESPACES DE VARIABLES  
%%%%%%%%%%%%%%%%%%%%%%%%%%%%%%%%%%%%%%%%%%%%

%pour N,Z,R 
\newcommand {\NN}{\mathbb{N}}
\newcommand {\NNe}{\NN^*}
\newcommand {\R}{\mathbb{R}}
\newcommand {\Rp}{\R_+}
\newcommand {\Rpe}{\R^*_+}
\renewcommand {\Re}{\R^*}
\newcommand {\Z}{\mathbb{Z}}


%%%%%%%%%%%%%%%%%%%%%%%%%%%%%%%%%%%%%%%%%%%%
% LES NOUVEAUX OPERATEURS  
%%%%%%%%%%%%%%%%%%%%%%%%%%%%%%%%%%%%%%%%%%%%

%optim
\DeclareMathOperator*{\argmin}{arg\,min}
\DeclareMathOperator*{\argmax}{arg\,max}

%analyse cvx
\DeclareMathOperator*{\conv}{conv}
\DeclareMathOperator*{\extr}{extr}
\DeclareMathOperator*{\vect}{vect}
\DeclareMathOperator*{\aff}{aff}





%%%%%%%%%%%%%%%%%%%%%%%%%%%%%%%%%%%%%%%%%%%%
% ICONES ET SYMBOLES HORS MATHS 
%%%%%%%%%%%%%%%%%%%%%%%%%%%%%%%%%%%%%%%%%%%%
\newcommand{\flch}{\item[$\rightarrow$]}

\newcommand{\attention}{\dbend}

\newcommand{\alamain}{{\LARGE\ding{45}}}

\newcommand{\surmachine}{
\begin{tikzpicture}[scale=0.1]
\draw[line width=1.2pt](-2,0) rectangle (2,3);
\fill(-1,0)--(1,0)--(2,-1)--(-2,-1)--(-1,0);
\end{tikzpicture}
}

\newcommand{\pasapas}{
\begin{tikzpicture}[scale=0.1]
\draw (0,0)--(0,1)--(1,1)--(1,2)--(2,2)--(2,3)--(3,3)--(3,4)--(4,4)--(5,4)--(5,0)--(0,0);
\end{tikzpicture}
}

\newcommand{\afaire}{
\begin{tikzpicture}[scale=0.1]
\draw (0,0)rectangle(4,4);
\end{tikzpicture}
}

\newcommand{\uneetoile}{{\LARGE$*$}\,}
\newcommand{\deuxetoiles}{{\LARGE$*\,*$}\,}
\newcommand{\troisetoiles}{{\LARGE$_*^*$}{\large$*$}\,}

\newtcolorbox{coder}{
enhanced,
boxrule=0pt,frame hidden,
borderline west={4pt}{0pt}{briqueRouge},
colback=black!3!white,
sharp corners
}
\newtcolorbox{codeb}{
enhanced,
boxrule=0pt,frame hidden,
borderline west={4pt}{0pt}{vertdEau},
colback=black!3!white,
sharp corners
}